\subsubsection{Variedades Diferenciais}

\paragraph{Definições}

\begin{framed}
\textbf{\textbf{Definition}: Produto wegde}\\
Uma base para as $k$-formas em uma variedade $d$ dimensional é dada por:
\begin{equation}
dx^{i_1}\wedge dx^{i_2}\wedge\cdots\wedge dx^{i_k} = \mathrm{sgn}(i) dx^{1}\wedge\cdots\wedge dx^{k}
\end{equation}
\end{framed}

\begin{framed}
\textbf{\textbf{Definition}: Derivada Exterior}\\
Seja $\omega$ uma  $k$-formas em uma variedade $d$ dimensional é dada por:
\begin{equation}
d\omega = \partial_\mu \omega_{i_1\cdots i_k} dx^{\mu}\wedge dx^{i_1}\wedge dx^{i_2}\wedge\cdots\wedge dx^{i_k}
\end{equation}
\end{framed}

\begin{framed}
\textbf{\textbf{Definition}: Dual de Hodge}\\
Seja $\omega$ uma  $k$-formas em uma variedade $d$ dimensional é dada por:
\begin{equation}
\star\omega = \frac{\sqrt{g}}{(d-k)!} g^{i_1 j_1} \cdots g^{i_k j_k} \epsilon_{j_1 j_2 \cdots j_d }  \omega_{i_1\cdots i_k} dx^{i_{k+1}}\wedge\cdots\wedge dx^{j_n}
\end{equation}
\end{framed}

\begin{framed}
\textbf{\textbf{Definition}: Coderivada}\\
\begin{equation}
d^\dagger = (-1)^{n(k+1)+1}\star d \star
\end{equation}
\end{framed}

\paragraph{Problemas}

\begin{framed}
\textbf{Problema: \textbf{Dual de Hodge}}\\
\begin{itemize}
\item Mostre que:
\begin{equation}
\star \star \omega = (-1)^{(k(n-k))} \omega, \quad \forall \omega \in \Omega_k
\end{equation}


\item Mostre que:
\end{itemize}

\begin{equation}
d^\dagger = (-1)^{k}\star^{-1} d \star
\end{equation}
\end{framed}

\begin{framed}
\textbf{Problema: \textbf{levi civita}}\\
\begin{itemize}
\item Sejam $x^\mu$ e $x'^\nu$ dois sistemas de coordenadas. Mostre que:
\end{itemize}

\begin{equation}
\epsilon_{j_1 j_2 \cdots j_d }` = J \epsilon_{j_1 j_2 \cdots j_d }, \quad \mathrm{onde} \\
J = \det \frac{\partial x^\nu}{\partial x'^\nu}
\end{equation}

\begin{itemize}
\item Mostre que:
\end{itemize}

\begin{equation}
g' = J^2 g
\end{equation}

\begin{itemize}
\item Mostre que $\sqrt|{g} \epsilon_{j_1 j_2 \cdots j_d }$ é invariante.


\item Mostre que:
\end{itemize}

\begin{equation}
\epsilon^{j_1 j_2 \cdots j_d } = \frac{1}{\sqrt{g}} \epsilon^{j_1 j_2 \cdots j_d }
\end{equation}
\end{framed}

\begin{framed}
\textbf{Problema: \textbf{Divergência}}\\
\begin{itemize}
\item Calcule $\star d \star omega$ em componentes, para uma 1-forma em 3 dimensões.
\item A partir do resultado acima, escreva asequencia de operações necessárias para escrever o divergente de um vetor em três dimensões.
\item Considerando um sistema de coordenadas ortogonais, escreva a expressão para o divergente de um vetor na base ortonormal.
\item Considerando coordenadas cilíndricas, escreva a expressão para o divergente.
\item Considerando coordenadas esféricas, escreva a expressão para o divergente.
\end{itemize}
\end{framed}

\begin{framed}
\textbf{Problema: \textbf{Pull Back}}\\
Sejam $M$ e $N$ variedades diferenciais de dimensões $m$ e $n$ respectivamente. Suponha $g: M\mapsto N$ uma aplicação suave. A aplicação $g^\star: \Omega(N) \mapsto \Omega(M)$ que mapeia as formas de $N$ nas formas de $M$ é chamada de \textbf{pullback} de $g$ e satisfaz:

\href{http://1.Se}{1.Se} $f: N \mapsto \mathrm{R}$ é função sobre $N$:

\begin{equation}
g^\star f = f \circ g
\end{equation}

\href{http://2.Se}{2.Se} $\eta, \omega \in \Omega_p(N)$:

\begin{equation}
g^\star (\eta \wedge \omega) = (g^\star \eta) (g^\star \omega)
\end{equation}

\href{http://3.Se}{3.Se} $\omega \in \Omega_p(N)$:

\begin{equation}
g^\star (d\omega) = d(g^\star \omega)
\end{equation}

\begin{itemize}
\item Procure se convencer de que a primeira propriedade define o pullback das 0-formas.


\item Considerando coordenadas $y^i$ em $N$ e $x^i$ em $M$, se convença de que:
\end{itemize}

\begin{equation}
g^\star (dy^i) (x) = d(y^i\circ g)(x) \Rightarrow \\
g^\star \omega(x) = \omega(g(x))_{i_1\cdots i_k} dg^{i_1}\wedge \cdots dg^{i_k}
\end{equation}

\begin{enumerate}
\item Agora, demonstre que esta definição satisfaz a propriedade 2.
\item Demonstre que esta definição satisfaz a propriedade 3.
\item COnsidere o caso em que $M$ e $N$ são idênticos e considere o pullback atuando na forma de volume. Qual é o resultado?
\item Seja $U = (0, \infty) \times (0, 2\pi)$, $V=\mathbb{R}^2 -{\textrm{eixo x não negativo}}$. Use coordenadas $(r, \theta)$ para $U$ e $(x, y)$ para $V$. Considere a aplicação $g(r, \theta) = (r\cos\theta, r\sin\theta)$. Considere $h=g^{ -1}$. Finalmente, sobre $V$ considere a forma $\omega = e^{x^2 + y^2} dx \wedge dy$:\begin{enumerate}
\item Calcule $g^\star(x), g^\star(y), g^\star(dx), g^\star(dy), g^\star(dx \wedge dy), g^\star \omega$.
\item Calcule $h^\star(r), h^\star(\theta), h^\star(dr), h^\star(d\theta)$.
\end{enumerate}
\end{enumerate}
\end{framed}