\subsubsection{Sobre este Livro}

\paragraph{Introdução}

\paragraph{Estrutura do Curso}

\begin{framed}
\textbf{Ementa}\\
Equações de Maxwell; Eletrostática e Magnetostática; Problemas de Condições de Contorno; Dielétricos; Ondas Eletromagnéticas Planas; Guias de Onda; Cavidades Ressonantes; Radiação e Antenas.
\end{framed}

\begin{framed}
\textbf{Bibliografia}\\
\begin{itemize}
\item \href{https://www.damtp.cam.ac.uk/user/tong/em.html}{notas de aula de David Tong - DAMTP/Cambridge}\begin{itemize}
\item Excelentes notas de aula disponíveis online. Usaremos alguns materiais de sua apresentação, em particular a parte sobre Eletromagnetismo e Relatividade.
\end{itemize}


\item \href{https://farside.ph.utexas.edu/teaching/jk1/jk1.html}{notas de aula de Richard Fitzpatrick - UT/Austin}\begin{itemize}
\item Excelente texto para um curso clássico de eletromagnetismo. Esta será a abordagem seguida para eletrostática e magnetostática.
\end{itemize}


\item J. David Jackson, ``Classical Electrodynamics''\begin{itemize}
\item Excepcional e denso texto clássico, do qual usaremos alguns tópicos selecionados em radiação.
\end{itemize}


\item A. Zangwill, ``Modern Electrodynamics''\begin{itemize}
\item Referência com apresentação moderna e bom nível de rigor. Utilizaremos alguns tópicos específicos e todo material do curso pode ser seguido por aqui.
\end{itemize}


\item Feynman, Leighton and Sands, ``The Feynman Lectures on Physics, Volume II''\begin{itemize}
\item Apresentação muito bonita de conceitos físicos e aplicações interessantes. Material de referência.
\end{itemize}


\item S. Cahn, B. Nadgorny, A Guide to Physics Problems - Part 1.\begin{itemize}
\item Diversos exercícios e soluções coletados de exames de qualificação pelo mundo.
\end{itemize}
\end{itemize}
\end{framed}

\begin{framed}
\textbf{Programa}\\
\begin{enumerate}
\item Unidade: Preliminares Matemáticos\begin{enumerate}
\item Geometria Diferencial\begin{enumerate}
\item Vetores e Tensores
\item Tensores isotrópicos
\item Coordenadas Ortogonais
\item Interlúdio: formas diferenciais
\item Operadores Diferenciais
\item Operadores Diferenciais em 3 dimensões
\item Teorema de stokes para formas diferenciais
\end{enumerate}


\item Análise Matemática\begin{enumerate}
\item Distribuições
\item Teorema de Helmholtz
\item Função de Green, Problema de Sturm Liouville e Transformadas Integrais.
\item Laplaciano em Coordenadas Esféricas
\end{enumerate}
\end{enumerate}


\item Unidade: A Estrutura do Eletromagnetismo\begin{enumerate}
\item Introdução\begin{enumerate}
\item As equações clássicas de Maxwell
\item A equação de onda
\item Potenciais Eletromagnéticos
\item Equação de Conservação de Carga
\item Escolhas de Gauge.
\end{enumerate}


\item Simetrias\begin{enumerate}
\item Invariância de Lorentz e Relatividade
\item Formulação covariante
\item Invariância de Gauge
\item Princípio da Ação
\item Acoplamento mínimo
\item Modelos sigma em 1 dimensão
\item Tensor momento-energia.
\item Momento Angular.
\end{enumerate}
\end{enumerate}


\item Eletrostática
\item Magnetostática
\item Campos Quase-Estáticos
\item Radiação
\end{enumerate}
\end{framed}